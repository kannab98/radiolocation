% Тип документа
\documentclass[a4paper,14pt]{extarticle}
\usepackage[utf8]{inputenc}
\usepackage[russian]{babel}
\usepackage[T2A]{fontenc}
\usepackage
    { % Дополнения Американского математического общества (AMS)
        amssymb,
        amsfonts,
        amsmath,
        amsthm,
        % Пакет для физических текстов
        physics,
        color,
        float,
        ulem,
        esint,
        esdiff,
        % 
    } 
    
\usepackage{mathtools}
\mathtoolsset{showonlyrefs=true} 

\usepackage{xcolor}
\usepackage{hyperref}
 % Цвета для гиперссылок
\definecolor{linkcolor}{HTML}{000000} % цвет ссылок
\definecolor{urlcolor}{HTML}{799B03} % цвет гиперссылок
 
\hypersetup{linkcolor=linkcolor,urlcolor=urlcolor,colorlinks=true}
\hypersetup{citecolor=linkcolor}
\hypersetup{pageanchor=false}

% Увеличенный межстрочный интервал, французские пробелы
\linespread{1.3} 
\frenchspacing 

\newcommand{\mean}[1]{\langle#1\rangle}
\newcommand*{\const}{\mathrm{const}}
\renewcommand*{\arctg}{arctg}
%\renewcommand*{\kappa}{\kappa}
\renewcommand*{\phi}{\varphi}

\newcommand{\tK}{\widetilde K}
%\renewcommand{\qty}{ }


\begin{document}


\section{Метод наименьших квадратов Прони}%
\label{sec:metod_naimen_shikh_kvadratov_proni}
\subsection{Матрица ковариации}%
\label{sub:matritsa_kovariatsii}

\begin{equation}
    \label{eq:Rxx1}
    R_{xx} = 
        \begin{bmatrix}
            r_{xx}[0] & r_{xx}^*[1] & \dots &  r_{xx}^*[n] \\
            r_{xx}[1] & r_{xx}[0] & \dots &  r_{xx}^*[n-1] \\
            \dots  & \dots    & \dots & \dots   \\ 
            r_{xx}[p] & r_{xx}[p-1] & \dots &  r_{xx}^*[n-p] \\
        \end{bmatrix}
\end{equation}
Можно обойтись без комплексного сопряжения, если учесть, что функция ковариации
обладает следующим свойством $r_{xx}[-p] = r_{xx}^*[p]$.

\begin{equation}
    \label{eq:Rxx2}
    R_{xx} = 
        \begin{bmatrix}
            r_{xx}[0] & r_{xx}[1] & \dots &  r_{xx}[-n] \\
            r_{xx}[1] & r_{xx}[0] & \dots &  r_{xx}[-n+1] \\
            \dots  & \dots    & \dots & \dots   \\ 
            r_{xx}[p] & r_{xx}[p-1] & \dots &  r_{xx}[-n+p] \\
        \end{bmatrix}
\end{equation}

Корреляция $r_{xx}$ вычисляется следующим образом, если индексс отсчитывать от
единицы:
\begin{equation}
    \label{eq:rxx1}
    r_{xx}[p] = \frac{1}{n-p+1} \sum\limits_{j=1}^{n-p} x[j+p] \cdot x^*[j]
\end{equation}
Или, если применять индексы $p<0$:
\begin{equation}
    \label{eq:rxx2}
    r_{xx}[p] = \frac{1}{n - |p| + 1} \sum\limits_{j=1}^{n-|p|} x^*[j+|p|] \cdot x[j]
\end{equation}

Зная автокорреляционную матрицу, можно найти коэффициент авторегрессии, решая
уравненния Юла-Уокера
\begin{equation}
    \label{eq:Rxx2}
        \begin{pmatrix}
            r_{xx}[0] & r_{xx}[-1] & \dots &  r_{xx}[-n] \\
            r_{xx}[1] & r_{xx}[0] & \dots &  r_{xx}[-n+1] \\
            \dots  & \dots    & \dots & \dots   \\ 
            r_{xx}[p] & r_{xx}[p-1] & \dots &  r_{xx}[-n+p] \\
        \end{pmatrix}
        \cdot 
        \begin{pmatrix}
            1 \\
            a_p[1]\\
            \vdots\\
            a_p[n]
        \end{pmatrix}
        =
        \begin{pmatrix}
            \rho_0 \\
            0\\
            \vdots\\
            0
        \end{pmatrix}
\end{equation}

\subsection{Оценка ошибки предсказания}%
\label{sub:otsenka_oshibki_predskazaniia}


Коэффициенты авторегрессии нужны для нахождения коэффициентов отражения $k_p$:
\begin{equation}
    \label{eq:kp}
    k_p = - \frac{
                \sum\limits_{n=1}^{\infty} a_{p-1}[n] r_{xx}[p-n]
        }{
                \rho_{p-1}
        }, \text{ где}
\end{equation}

дисперсия (?) $\rho_p$ связана рекурсивным соотношением с $\rho_0 = r_{xx}[0]$
\begin{equation}
    \label{eq:}
    \rho_p = \rho_{p-1} (1 - \abs{k_p}^2)
\end{equation}

Теперь можем найти ошибку предсказания вперед и назад
\begin{align}
    \label{eq:err-front}
    e_p^f[n]  =  e_{p-1}^f[n]    + k_p e_{p-1}^b [n-1]\\
    \label{eq:err-back}
    e_p^b[n]  =  e_{p-1}^b[n-1]  + k_p^* e_{p-1}^f [n]\\
\end{align}









\end{document}
